Lyra-Network Position Paper for \textbf{W3C Workshop on Web Payments}.

Authors:

Grégory Estrade
\href{mailto:gregory.estrade@lyra-network.com}{gregory.estrade@lyra-network.com}

Laurent Penou
\href{mailto:laurent.penou@lyra-network.com}{laurent.penou@lyra-network.com}

\section{Introduction}\label{introduction}

\begin{quote}
``The web is more a social creation than a technical one. I designed it
for a social effect --- to help people work together --- and not as a
technical toy. The ultimate goal of the Web is to support and improve
our weblike existence in the world. We clump into families,
associations, and companies. We develop trust across the miles and
distrust around the corner.''
\end{quote}

\emph{Tim Berners-Lee, ``Weaving the Web''}

\begin{quote}
``I can't in good conscience allow the U.S. government to destroy
privacy, internet freedom and basic liberties for people around the
world with this massive surveillance machine they're secretly
building.''
\end{quote}

\emph{Edward Snowden}

Allow us to be a bit provocative: all card-based payment schemes are
broken. They basically fail at understanding the need of both identity
and anonymity, which are the critical topics that should be addressed at
the Web level. Recent issues highlighted in the news about privacy,
stories about sensitive data theft and global surveillance, should alert
us, as citizens of the world, to how we want the future of information
processing to be.

We need to get rid of insufficient data protection schemes, avoid
whenever possible the use of trust models relying on a single central
authority. The problem is global, and is way beyond the scope of this
document, but we still can do something about it for the subject
addressed in this workshop.

Our main concern is to find ways to enforce good practices, by
participating in the definition of a set of standards that would allow
seamless introduction of distributed, networked trust models, as we can
find nowadays in cryptocurrencies schemes like Bitcoin or Dogecoin.

As we are also pragmatic, and aware that such changes will take years,
we will address existing payment scenarios, and hopefully pave the way
for safer traditional schemes.

\section{Digital wallets and distributed models
generalization}\label{digital-wallets-and-distributed-models-generalization}

In this chapter, we will conduct a tour of the most common payment
scenarios.

Then we will make some proposals to provide a more uniform payment
processing scheme, so that the user experience feels more comfortable,
and more trustworthy as well, by limiting disclosure of required
information to zones where it becomes necessary.

Based on this work, we will envision some work-arounds for exisitng
payment methods, and discuss how Web standards could accelerate the
migration.

\subsection{Card processing}\label{card-processing}

\subsubsection{State of the art}\label{state-of-the-art}

Let's review the steps of a Visa/MasterCard e-commerce payment:

\begin{itemize}
\item
  The cardholder provides his card details to the acceptor.
\item
  These details, along with transaction information (amount, currency)
  are relayed to the issuer through the acceptor for 3-D Secure
  authentication.
\item
  As a whole, the same information is relayed through the acceptor and
  the acquirer to the issuer for authorization.
\end{itemize}

The first step assumes that the cardholder is ready to trust the
acceptor at some level.

The second step which is ultimately important, as it is designed to
provide a strong authentication of the cardholder, can be bypassed.

Choosing not to use 3-D Secure is often a decision of the acceptor, that
has to put in balance the risk of a void sale, compared to the risk of a
fradulent transaction. Again, this is a matter of trust level.

Finally, the third step feels rendundant with the second step,
information-wise.

The first step is basically \emph{flawed at Web scale}.

Althought it has its justifications for retail transactions processing,
on which the e-commerce model is based, this step feels unnecessary and
rises many security concerns.

\subsubsection{Proposed alternative}\label{proposed-alternative}

Another way to envision this payment could be:

\begin{itemize}
\item
  The acceptor provides some identification information, as well as
  transaction details to the cardholder.
\item
  The cardholder authenticates himself to the issuer and provides
  transaction details.
\item
  Authorization is performed at issuer level, and its result is handed
  back to the acceptor through the acquirer.
\end{itemize}

The advantages of this scheme, besides the fact that it feels more
natural, is that the trust environments are already there, as the
cardholder only has to trust the issuer, and the same applies to the
acceptor and the acquirer.

\subsubsection{Dealing with legacy}\label{dealing-with-legacy}

We believe that these suggested changes, although they seem important,
could be handled more easily with the help of new standards and their
implementation at the browser level, while keeping all the 3-D Secure
infrastructure and authentication scheme. However, this is beyond the
scope of this abstract.

\subsection{Asymmetry concerns}\label{asymmetry-concerns}

Running an e-commerce business still requires in most cases the
subscription of an acceptor contract and a dedicated account for card
processing (with the notable exception of PayPal).

Those make the process of receiving money through most debit and credit
cards a complicated one, compared to the ease of subscription and use of
these cards.

Digital currencies don't suffer from these issues, and it can be stated
that perfect symmetry is not only built-in, but also a requirement for
them to work.

Lately, initiatives have come to light, as some markets are emerging,
and the focus of established financial institutions on small businesses
and individuals makes us envision a unprecedented growth in payments.

E-commerce payments will benefit from this growth, for sure, but the
higher increase rate should happen for retail, as the mobile
point-of-sale (mPOS) solutions allow the same category of merchants to
accept cards.

To sum up, what can be foreseen about Web payments is that in years to
come, anyone may be concerned by them not only as a customer, but as a
seller as well.

\section{Beyond the wallet scheme}\label{beyond-the-wallet-scheme}

Users of digital currencies are familiar with the idea of digital
wallets. These wallets can be used for both selling and purchasing goods
and services, and are based on public key cryptography.

Our proposal is to extend the same wallet concept to include both
existing payment methods and acceptance means, providing a single local
entity to manage both centralized and distributed schemes.

Such a wallet could be shared among different devices: desktops,
laptops, smartphones, tablets, etc. but also on servers.

Depending on the use cases of the public keys, they may be signed by a
Certificate Authority, or using a ``Web of trust'' scheme instead.

\subsection{Use case}\label{use-case}

\subsubsection{Initialization}\label{initialization}

From the customer (cardholder) side:

\begin{itemize}
\item
  Alice owns an account at Bank A, which also provided her with a credit
  card.
\item
  Alice provided a public key to Bank A, and after some identity
  verification, Bank A signed this public key.
\item
  Alice also owns some coupons from Company C, and owns a public key
  signed by the same entity.
\end{itemize}

From the merchant (acceptor) side:

\begin{itemize}
\item
  Bob runs an e-commerce store. Bank B provided him with an account, as
  well as an acceptor contract.
\item
  Bob provided a public key to his bank, and after some identity
  verification, Bank B signed this public key.
\item
  Bob's store accepts coupons from Company C, therefore one of Bob's
  public key has been signed by the same entity.
\end{itemize}

\subsubsection{Payment process}\label{payment-process}

Alice shops at Bob's store. On checkout, Bob's server provides a set of
public keys related to his acceptance capabilities, as well as list of
accepted payment methods, an amount, a currency, and a transaction
identifier.

Alice's browser displays the available payment methods providers, namely
Company C and Bank A.

Alice chooses Company C. The coupon itself is stored in the wallet, so
it could be handed back directly to Bob's store.

Alice then chooses Bank A. Bank A identifies Alice, due to her signed
public key. An additional authentication step is then performed by Bank
A. Upon success, Alice chooses her credit card.

Bank A performs the authorization based on information provided by Bob's
store (this information is digitally signed by Bob's private key).

Bank A relays the authorization result to Bank B, which itself relays
the result to Bob's server.

\subsection{Security and Identity
Management}\label{security-and-identity-management}

In a previous paragraph, we stated that some public keys may be signed
by a Certificate Authority managed by a financial institution, be it an
acquirer or an issuer.

These public keys might be used for \emph{identification} purposes, from
the financial institution side. However, from the cardholder/acceptor
side, these are used for \emph{anonymization} purposes.

Even in distributed payment schemes as provided by digital currencies,
these public keys should be used solely for \emph{anonymization}
purposes.

However, at some point, there is a need for strong identification, in
order to address the following subjects:

\begin{itemize}
\item
  Distributed identity management.
\item
  Secure transportation of the aforementioned private keys across
  devices.
\item
  Revocation of these keys.
\end{itemize}

We feel that this specific part of the whole architecture is probably
the most important and difficult one.

However, some standards are already there, to help build up its
foundations, namely the WebID protocol and FOAF specification.

We also believe that the specifications issued by the JOSE working
group, especially those related to key protection, should be taken into
account.

We would like to address the issue with local storage of private keys,
as it is a ``single point of failure'', and envision a distributed
approach on that subject.

\section{References}\label{references}

Distributed private key management:
\url{http://individual.utoronto.ca/aldar/paper/2012/dpkg.pdf}

WebCryptoAPI: \url{http://www.w3.org/TR/WebCryptoAPI/}

Web of trust: \url{http://en.wikipedia.org/wiki/Web_of_trust}

Protecting JSON Web Key (JWK) Objects:
\url{http://tools.ietf.org/html/draft-miller-jose-jwe-protected-jwk-02}

WebID provider using Node.js:
\url{http://magnetik.github.io/node-webid-report/}

FOAF Vocabulary Specification 0.99: \url{http://xmlns.com/foaf/spec/}

WebID 1.0: \url{http://www.w3.org/2005/Incubator/webid/spec/}
